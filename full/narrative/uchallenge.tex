

% Unique challenges to accomplishing the work and planned mitigations.

Like RANS, one can expect the success of ROMs for turbulent flows to
be somewhat sporadic.  
  Unlike elliptic PDEs, for which there is robust and well-established
stability and convergence theory, the nonsymmetric nonlinear NSE
largely defy such analysis (\cite{fick18,patera05}, however, make progress in this
direction), which makes ROM-for-turbulence development challenging.
Like RANS, however, the results can be spectacularly rewarding when ROMs
work---thousands of convective time units simulated per minute for small $N$.
   Close inspection of the mathematically-driven ROM literature for LES and
RANS reveals that many of the examples do not exhibit emergent turbulence.
Turbulent cases where ROMs do succeed often feature clear time-scale separation
(e.g., 3D flow past a cylinder \cite{wells2017evolve} or a precessing jet
\cite{akkari19}), which indicates that succcess is more likely when there 
is an emergent low-rank dynamical system.  If ROMs work only in these
cases, we could still consider this a success.

Fortunately, we have a team that is well-versed in multiple numerical methods
and their applicability across the broad reactor-TH landscape.  The team also
understands the benefit/costs of TH analysis in this application space.  It is
a team that cannot afford to shrink from tackling challenging problems.  As
such, we expect to push the boundaries of this multiscale approach to turbulent 
TH applications.
