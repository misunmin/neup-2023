

% Unique challenges to accomplishing the work and planned mitigations.

The most significant challenge is to ensure that the analysis costs
are low enough to be useful as a design tool that can be used on a
workstation.  The $O(N^3)$ storage and work complexity puts pressure on
delivering an accurate solution with a relatively small basis set.  As noted,
our strategy makes significant strides in this direction.  However, 
complex turbulent flows have high-dimensional attractors that might not
be adequately captured with a practical-sized basis set.  Several
mitigation strategies have been proposed in the literature that are
compatible with the proposed approach.  One of these is to use a POD
based on a subset of the FOM computational domain, where the solution
could be approximated by a lower number of modes.   
    One could also apply the pMOR strategy to the unsteady
RANS equations, which have significantly lower-dimensional attractors
than their high-Reynolds number DNS and LES counterparts.  
The presence of eddy viscosity/diffusivity in these models leads to
a high degree of nonlinearity that is greater than the quadratic
nonlinearity of the primitive equations.  A first step, however, would
be to attempt a linearization of the RANS coefficients so that the
overall complexity of the pMOR does not exceed $O(N^3)$.
  Another cost-saving strategy is to parallelize the {\em online} portion of
the ROM (\ref{eq:const}), which is fairly straightforward for a handful of
processors.  Further reduction of {\em offline} costs could also be effected
by exploiting available parallelism in evaluation of the 3rd-order tensor.
   Finally, identifying rigorous error-estimators will be a 
significant challenge but much progress has been made towards
error estimators for the NSE \cite{fick18,patera05}.
