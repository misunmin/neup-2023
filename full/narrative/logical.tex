
%\newpage
\vspace*{.08in}\noindent \underline{\textbf{ 
              Logical Path to Accomplishing Scope.}} \\[-2ex]

%% Logical path to accomplishing scope, including descriptions of tasks. This
%% section will provide a clear, concise statement of the specific objectives/aims
%% of the proposed project. This section should be formatted to address each of
%% the merit review criterion and sub- criterion listed in Part V, Section A.
%% Provide sufficient information so that reviewers will be able to evaluate the
%% application in accordance with these merit review criteria. DOE has the right
%% to evaluate and consider only those applications that separately address each
%% of the merit review criteria.

\input narrative/tasks
In the sequel we describe subtasks to be addressed in support of the
main Tasks outlined above.
\\[-2ex]

\input narrative/challenge

\input narrative/stabilized

\input narrative/coupled

% \textbf{PAUL: this is old stuff.}
% The effort requires correct problem specifications (parameters, boundary
% conditions, mesh resolution, etc.) and efficient set-up of the large-scale
% full-order models (FOMs) on DOE's leadership computing facilities.  We
% anticipate acquiring computer time through DOE's standard award mechanisms
% (director's discretionary, ALCC, or INCITE) and leveraging other resources
% already available to the team members.  The ROM-modeling will involve extension
% of the 2D unsteady Navier-Stokes setting in \cite{fick18} to 3D turbulent flows
% with energy transport.  The error estimators will be modified to account for
% quantities of interest (QOIs) relevant to the upper plenum flow, such as mean
% and RMS temperature distributions, mean outflow distributions, etc.
% Insight to temporal behavior will also be accessible from visualizations
% of the ROM reconstructions.

% \input narrative/plan
% \input narrative/rom

