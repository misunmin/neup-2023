
%% • Relevance and Outcomes/Impacts: This section will explain the
%%   program relevance/priority of the effort to the objectives in the
%%   program announcement and the expected outcomes and/or impacts.


The methods developed in this proposal aim to deliver a tool for accurate and
computationally inexpensive prediction of key thermal-hydraulic phenomena in
liquid metal reactors. The operation, safety, and design of Sodium fast
reactors are impacted by several critical factors, including thermal
stratification, mixing in large enclosures such as liquid metal pools, and
mixed convection. The lack of understanding and adequate modeling capability
for these phenomena leads to high uncertainty, which limits the reactor's
overall performance by requiring excessive margins. The proposed methods have
the potential to help reduce these margins, leading to better economics. 

During loss of flow incidents, low flow conditions at the periphery of the
core, along with steep radial core gradients, can result in mixed convection.
This impacts the thermal expansion of assembly ducts and the overall
thermo-mechanical behavior of the core, which is a crucial reactivity feedback
during these events.  Thermal stratification can occur during loss of flow
incidents, potentially hindering the establishment of natural circulation,
which is crucial during these events. Furthermore, Inadequate mixing in pools
contributes to thermal stratification and can result in fluctuating thermal
loads (thermal striping) on structural components during normal operation,
leading to fatigue and related failures.

These phenomena affect the reactor's thermal limits and margins, with similar
considerations applicable to all liquid metal reactors. We note that a lack of
cost-effective and accurate prediction methods for these phenomena has long
been recognized as a limitation in fast reactor gap analysis studies.



%%  The methods developed in this proposal aim to deliver a tool for accurate
%%  low-cost prediction of key thermal-hydraulic phenomena in liquid metal reactors.
%%  The operation, safety, and design of Sodium fast reactors are impacted by
%%  several critical factors, including thermal stratification, mixing in large
%%  enclosures such as liquid metal pools, and mixed convection. The lack of
%%  understanding and adequate modeling capability for these phenomena leads to
%%  high uncertainty, which limits the reactor's overall performance by requiring
%%  excessive margins. The proposed methods have the potential to reduce these
%%  margins, leading to better economics. 
%%  
%%  Thermal stratification can occur during loss of flow incidents in the upper
%%  plenum, potentially hindering the establishment of natural circulation, which
%%  is crucial during these events. Inadequate mixing of the fuel assembly flows in
%%  the upper plenum contributes to thermal stratification and can result in
%%  fluctuating thermal loads on structural components during normal operation,
%%  leading to fatigue and related failures. During loss of flow incidents, low
%%  flow conditions at the periphery of the core, along with steep radial core
%%  gradients, can result in mixed convection. This impacts the thermal expansion
%%  of assembly ducts and the overall thermo-mechanical behavior of the core, which
%%  is a crucial reactivity feedback during these events. These phenomena affect
%%  the reactor's thermal limits and margins, with similar considerations
%%  applicable to all liquid metal reactors. The lack of cost-effective and
%%  accurate prediction methods for these phenomena has long been recognized as
%%  a limitation in fast reactor gap analysis studies.
%%  
%%  
%%  %%%    Thermal stratification can occur during loss of flow incidents in the upper
%%  %%%  plenum, potentially hindering the establishment of natural circulation which is
%%  %%%  crucial during these events.
%%  %%%  Inadequate mixing of the fuel assembly flows in the upper plenum contributes to
%%  %%%  thermal stratification and can result in fluctuating thermal loads on
%%  %%%  structural components during normal operation, leading to fatigue and related
%%  %%%  failures.
%%  %%%    During loss of flow incidents, low flow conditions at the periphery of the
%%  %%%  core, along with steep radial core gradients, can result in mixed convection.
%%  %%%  This impacts the thermal expansion of assembly ducts and the overall
%%  %%%  thermo-mechanical behavior of the core, which is a crucial reactivity feedback
%%  %%%  during these events.
%%  %%%    All of these phenomena affect the thermal limits and margins of the reactor,
%%  %%%  with similar considerations applicable to all liquid metal reactors. A lack of
%%  %%%  cost-effective and accurate prediction methods for these phenomena has long
%%  %%%  been recognized as a limitation in fast reactor gap analysis studies.
%%  %%%  
%%  %%%  %%   \hspace{.2in} \begin{minipage}[t]{5.9in} \noindent \textbf{Old Text.}
%%  %%%  %%   Expected outcomes and impact of this project include:
%%  %%%  %%   \\[-4ex]
%%  %%%  %%   \begin{itemize}
%%  %%%  %%   \item % \textbf{i.}
%%  %%%  %%   A significant advance of the state-of-the-art of reduced-basis methods
%%  %%%  %%   for engineering analysis of turbulent flows;
%%  %%%  %%   \\[-4ex]
%%  %%%  %%   \item % \textbf{ii.}
%%  %%%  %%   A mechanism to leverage leadership computing facilities for advanced
%%  %%%  %%   engineering design and analysis of thermal-hydraulic systems;
%%  %%%  %%   \\[-4ex]
%%  %%%  %%   \item % \textbf{iii.}
%%  %%%  %%   A readily-used tool for for engineering analysis of turbulent heat transfer;
%%  %%%  %%   and
%%  %%%  %%   \\[-4ex]
%%  %%%  %%   \item % \textbf{iv.}
%%  %%%  %%   Analysis of several industrially-relevant applications.
%%  %%%  %%   \\[-4ex]
%%  %%%  %%   \end{itemize}
%%  %%%  %%   
%%  %%%  %%   %% The ultimate goal is to augment Nek5000 with a readily usable module for
%%  %%%  %%   %% reduced-order analysis that will significantly enhance the value of
%%  %%%  %%   %% high-end numerical simulations and that can be coupled directly with
%%  %%%  %%   %% system analysis modules.
%%  %%%  %%   
%%  %%%  %%   The proposed work will advance thermal-hydraulic design and analysis capabilities,
%%  %%%  %%   with an initial target focusing on mixing in the upper plenum.
%%  %%%  %%     Low-cost analysis tools are critical to efficient design cycles.   DOE's 
%%  %%%  %%   leadership computing facilities can provide detailed DNS/LES results at isolated
%%  %%%  %%   design points, but at great costs (months of runtime).  The aim of this project 
%%  %%%  %%   is to provide a reliable mechanism to translate detailed information from
%%  %%%  %%   full-order models into rapid analysis tools equipped with validated error
%%  %%%  %%   estimates.  By setting realistic goals of time-averaged QOIs and using an
%%  %%%  %%   $h$-adaptive training strategy, the proposed scheme has the potential to be
%%  %%%  %%   practical in highly turbulent configurations where standard POD-Galerkin
%%  %%%  %%   schemes fail.  The project will provide a straightforward interface that is as
%%  %%%  %%   easy to use as Nek5000 and so will provide a tool that is readily available to
%%  %%%  %%   the existing Nek5000 user base and as an input to SAMs in support of LMR design
%%  %%%  %%   and licensing.
%%  %%%  %%   \\
%%  %%%  %%   \end{minipage}
