LMR system analysis depends critically on low-cost simulations of the upper plenum response, which is frequently modeled as a perfectly-mixed 0-D volume. Mixing and thermal stratification can be simulated reasonably well with current high-fidelity approaches on DOE's leadership computing facilities but this approach is generally too expensive for design and analysis purposes. The objective of this project is to develop reduced-order models (ROMs) that will improve the accuracy of LMR system-level analysis with low overhead. These new models will systematically mine high-fidelity DNS, LES, or uRANS simulations to construct low-order dynamical systems that can be couple with a systems analysis code such as the SAM code being developed under NEAMS. The ROM will be developed as a new software module in the NEAMS-supported Nek5000 thermal-fluids code.

Novel contributions of the proposed ROM include a new constraint-based POD formulation that keeps the dynamical system close to the originating attractor and allows accurate tracking of the dynamics for relatively low-dimensional (and, hence, low-cost) models; an $h$-greedy training strategy that also leads to reduced model system sizes; and effective dual-norm-based error estimates that inexpensively indicate errors in the reduced model. The concepts will be extended from their current state to nuclear-engineering-relevant applications and will be integrated into the Nek5000 repository with a focus on ease-of-use both in Nek5000 and in SAM.

Verification and validation are built into the project. The development effort follows systematically from 2D convective transport to 3D turbulence, 3D convective transport, and culminates in buoyancy-driven 3D turbulent flow in an upper plenum model. At each stage, canonical fluid/thermal examples will be explored to validate both the Nek5000 simulation and the success of the ROM in reproducing quantities of interest. In parallel to the ROM development effort, a series of highly detailed LES simulations of flow in an upper plenum model will be performed and validated so that useful and tractable quantities of interest may be identified early in the project.  These simulations will provide useful data in their own right and results will be made available to the scientific community. 
