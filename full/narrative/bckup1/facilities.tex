
%% The principal compute costs for the project will be incurred by the large
%% offline DNS/LES computations performed with Nek5000.  The online computations
%% will be at workstation or laptop scales.  The pMOR training will require
%% online/offline iterations that will be most effectively performed at a
%% supercomputing center.

\vspace{.05in}

\noindent
The proposed computational work will be performed at various high performance
computing facilities, where the PIs already have been awarded allocations.

The PI has access to {\em Delta} at UIUC, which has 440 NVIDIA A100 GPUs.  His
lab also has several multi-CPU workstations, including one with two NVIDIA
Titan V GPUs for local development work.  Through the DOE ECP CEED project, he
and his students have access to DOE's pre-exascale and exascale platforms for
testing at scale.  He also frequently has access to ALCF and OLCF platforms
through DOE's standard allocation processes.

The co-PI at Penn State has an allocation at the local ROAR supercomputer that
will be leveraged for both testing and optimization simulations. ROAR is Penn
State’s flagship research cluster maintained by ICDS. Roar includes basic,
standard, and high memory compute as well as GPU processors. ROAR operates more
than 30,000 Basic, Standard and High Memory cores to support Penn State
research. The system provides dual 10- or 12-core Xeon E5-2680 processors for
Basic and Standard memory configurations and quad 10-core Xeon E7-4830
processors for High Memory configurations. In addition to access to ROAR,
request for allocations with DOE’s ALCC and INCITE programs will be submitted
for access to DOE’s supercomputing facilities if needed. Access to INL Sawtooth
will also be pursued.

The Nuclear Science and Engineering division at ANL hosts the Nek5k cluster,
which is dedicated for high-fidelity CFD simulations using Nek5000/NekRS. This
cluster includes 40 nodes, each with dual 20-core Xeon Gold 6230 processors and
92GB of memory connected by a high speed Infiniband network. A planned upgrade
will add two additional nodes, each with 8 Nvidia A100 GPUs, expected to be
available in early 2023.
\\
