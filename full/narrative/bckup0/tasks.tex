\noindent
\textbf{Proposed Tasks.}
We propose the following Tasks to realize a multiscale simulation
capability within Nek5000/RS/ROM for reactor analysis.
%% , following the schedule laid out in Fig.~\ref{fig:gantt}. 
\\[-1ex]

\noindent
\textbf{Task 1: Data Generation.} We will perform high fidelity simulations of 
the flow in SFR fuel assemblies (e.g., Fig. \ref{fig:sum}, left), including 
a range of steady-state conditions and a loss-of-flow transient scenario.
\\[-3ex]
\begin{description}
\item{$\; i.$}
Define benchmarks for SFR fuel assemblies. These will be used for validation.
\\[-4ex]
\item{$\, ii.$}
Develop data to construct ROMs for SFR fuel assemblies. This
task will support Task 2.  
\\[-4ex]
\item{\em iii.} Develop data to validate ROMs for both
transient and parameter sweeps to support Task 3.
\\[-3ex]
\end{description}
In Task 1, we will leverage several cases developed as benchmarks for mixed
convection and thermal striping as part of a recent IRP. The cases are
summarized in Table~\ref{tab:cases}. For this project, we will augment these
cases by performing simulations for a broader range of parameters and
collecting additional snapshots as needed. We remark that using well-defined
past benchmarks also allows us to evaluate the performance advantage of
ROM-based approaches. \\[-2ex]

\begin{table}
\centering
\begin{tabular}{|l|c|c|c|}
\hline
\textbf{Case} & \textbf{Size} (grid points) & \textbf{Run time} (CU) & \textbf{Snapshots} \\\hline
Mixed convection (19 pins) \cite{kraus22}& $1\cdot 10^{9}$ & 20,000 & 5000 \\
\hline
Mixed convection (61 pins) \cite{kraus22}& $3\cdot 10^{9}$ & 2000 &  5000 \\
\hline
Parallel Jets \cite{acierno22}& $1\cdot 10^{9}$ & 200,000 & 10,000 \\
\hline
\end{tabular}
\caption{\label{tab:cases} List of existing cases.}
\end{table}

% \begin{figure}[b!] \centering
%    {\setlength{\unitlength}{1.0in} \begin{picture}(6.5,2.500)(0.0,0)
%      \put(0.9,-.00){\includegraphics[height=2.7in]{figs/neup_gantt_v1.png}}
%    \end{picture}} 
%    \caption{Timeline of proposal by fiscal year.  \label{fig:gantt}
% \\[-2ex]
% }
% \end{figure}


\noindent
\textbf{Task 2: Algorithmic Development.} This is the core of the project.
We seek to develop advanced ROM methods that are seemlessly
extended to a multi-scale framework in Nek5000/RS/ROM.
\\[-3ex]
\begin{description}
\item{$\; i. \; \; \;$}
Continued development of ROM closure models to increase robustness 
for high Rayleigh/Reynolds flows.  This effort will build on earlier and 
ongoing work
\cite{kaneko22a,kaneko22,tsai22a,kaneko20a,mou2021}.
\\[-3ex]
\item{$\, ii. \;\;$}
Identify strategies to equip the ROM for long-time transients.  
Proposed extensions to the methods demonstrated in \cite{kaneko20a}
include error-indicated pMOR and improved stabilizing bases.
\\[-3ex]
\item{\em iii. \;}Develop multi-scale methods that couple ROMs with LES on a 
subset of the
domain.  Principal challenges are to support general or parameterized inflow
conditions for the ROM and to quantify relaxation distances over which small
scale structures will evolve near the ROM-LES interface. The goal will be to
combine reactor-scale ROM simulations with localized detailed simulations in
regions of interest. This leverages past work on rod bundles and sub-channels,
whose geometric properties have been exploited to maximize the use of small
domain high-fidelity simulation to improve RANS results in a multi-scale
fashion \cite{martinez2019a}.  
\\[-2ex]
\end{description}%

\noindent
\textbf{Task 3: Validation and Demonstration.} 
Here we validate and demonstrate the proposed methods.
\\[-3ex]
\begin{description}
\item{$\; i.\; \;$}
Validate ROM transient calculations with high-fidelity counterparts
in an SFR rod-bundle under low-flow conditions.
\\[-4ex]
\item{$\, ii. \,\;$}
Perform ROM-based parameter sweeps and compare these to corresponding
FOM results for validation.
\\[-4ex]
\item{\em iii.}
Demonstrate accelerated transient capability for cases that are currently not
feasible. We will implement the proposed ROM-based and multiscale models within
Cardinal \cite{cardinal}. This will enable their usage within the systems
analysis code SAM \cite{hu2021}. The co-PI at Penn State has been working
extensively on the integration of SAM and NekRS within Cardinal.
\\[-2ex]
\end{description}
