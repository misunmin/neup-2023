

In this {\em reproduction mode}, the velocity and (possibly) the temperature
field need not be resolved at their original $O(\cN)$ fidelity, given that the
basis functions $\bz(\bx)_n$ are smooth.  In Nek5000, it's relatively simple
to interpolate the basis functions onto coarser meshes ($\cN' < \cN$) by
using a lower-order polynomial in the underlying spectral element discretization
or using Nek5000's efficient (and spectrally accurate) grid-to-grid interpolation.
Conversely, if one wanted {\em higher} resolution for the thermal problem (e.g.,
in the case of high Prandtl or Schmidt numbers), the reproduction could be
evolved on a higher-order spectral element mesh.

This semi-reduced mode opens up several possibilites for parametric-space
exploration.  One recent idea is to use DNS/LES in the region of interest and
to simply use the ROM to drive the flow dynamics in the far-field region where
less detail might be required \cite{bergmann18}.  Such approaches are obviously
better suited to external aerodynami flows rather than internal flows typical
of reactors, but it might be applicable in large regions such as the upper
plenum.  That such an approach












