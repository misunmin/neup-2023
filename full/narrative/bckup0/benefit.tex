
\newpage

\subsection*{Benefit of Collaboration}

The project is a collaborative effort between computer scientists, applied
mathematicians, mechanical engineers, and nuclear engineers, comprising
extensive experience in CFD, reduced-order modeling, and reactor thermal
hydraulics analysis.    
  The PI and other investigators have a long experience in CFD simulations
using Nek5000 and have developed a strong partnership in the past years with
the code development group in Argonne National Laboratory's NE Division.  
Simulations include several designs and components of advanced reactors such as
LMR and HTGR. 


Collaborators from TAMU and Westinghouse are advanced users of all tools and
techniques proposed to successfully conduct and complete the required tasks.
These include:
\begin{itemize}
\item
Advanced use of Nek5000, which is to be used to perform the
high fidelity simulations of the selected geometries. 
\item
Extensive use of CFD coupled with system codes. TAMU has already coupled DAKOTA
with RELAP5-3D and STAR-CCM+ performed UQ and sensitivity analysis to support
reactor analysis for industrial applications. 
\item
Outstanding use of system-level codes for Light Water Reactors and Non-Light
Water Reactor applications, including coupled version system-CFD codes. TAMU
has also the SAM.
\end{itemize}
TAMU investigators have conducted an extensive experimental activity and
produced high resolution measurements of flows under natural circulation
conditions using advanced laser-based techniques. Data is fully available to
all investigators and will be used during the validation phase of the project. 
\\

Paul Fischer is a Blue Waters Professor of Computer Science and Mechanical
Science and Engineering at the University of Illinois.
He is the principal author of the open-source thermal-fluids
simulation code Nek5000 (originally developed as Nekton 2.0 by Patera,
Fischer, Ronquist and Ho) that is part of the NEAMS-supported SHARP code
suite.  Dr. Fischer is an expert in high-order numerical methods, scalable
parallel algorithms, and turbulent flow and thermal transport.  His research
group is focused on the development of advanced numerical algorithms for
turbulence simulations.  These algorithms are typically implemented within
Nek5000, which provides a scalable framework for the test and deployment
of the algorithms.  Ph.D. students supervised under this project
will have a unique opportunity to work on a project spanning from
rigorous mathematical theory to challenging industrial thermal-fluids
problems.   As part of the Blue Waters professorship, Dr. Fischer's
group has access to a large allocation on the NCSA Blue Waters platform,
which features 362,000 compute cores.
\\

Yassin A. Hassan is Professor and Department Head of the Department of
Nuclear Engineering at Texas A \& M University, and a joint appointment as
Professor of the Department of Mechanical Engineering. He is a Fellow in the
ANS and the American Society of Mechanical Engineers (ASME). He received the
2004 Thermal Hydraulics Technical Achievement Award of ANS; the George
Westinghouse Gold Medal of ASME. His research interests include computational
and experimental thermal hydraulics; reactor safety; two-phase flow;
turbulence and laser velocimetry and imaging techniques. His synergistic long
and proven experience in experimental and computation methods (ranging from
specialized system codes to CFD and advanced computational tool Nek5000) will
be crucial for the project success. 
\\


Milorad Dzodzo is an expert in thermal-hydraulics, computational and
experimental fluid mechanics and heat transfer.
  His expertise covers analytical approaches, numerical methods,
  and experimental measurements, which have been applied in the
  analysis of a variety of NE thermal-hydraulics applications including
  fuel assembly designs, primary coolant pumps, primary moisture separators, 
   pipe networks and flow straighteners.
  He and his group have been using Nek5000 for over five years and are
  currently working on the development and validation of advance turbulence
  models in Nek5000.
\\


Anthony Patera is the inventor of the spectral element method (Patera,
J. Comp. Phys. {\bf 54}, 1984) that forms the foundation of Nek5000.
  He has extensive experience in high-order methods, parallel computing,
heat transfer, and turbulence.
  He has been working in the area of model order reduction for over two
decades, with particular emphasis on rigorous error-bounded model order
reduction methods for partial differential equations.  
   The proposed reduced-order model builds on work initiated by his
group. 
\\

Tommaso Taddei has a background in applied mathematics and mechanical
engineering and is currently a post-doctoral associate in the mathematics
department at Paris VI.  His work is primarily in the area of model order
reduction.  He is the corresponding author on {\em A reduced basis technique
for long-time unsteady turbulent flows} submitted to J. Comp. Phys., 2017, 
which forms the basis for the approach developed in this project.  This paper
was co-authored with Patera and Fick (formerly of the TAMU group).
\\


Dr. Gert Van den Eynde is Head of the Nuclear Systems Physics Expert Group at
the Belgian Nuclear Research Center (SCK-CEN). This research group is
responsible for the neutronics and safety analysis of the existing BR1 and
VENUS reactors and the MYRRHA Accelerator Driven System under development. He
  Van den Eynde holds MS degrees in Computer Science Engineering and 
Nuclear Engineering and a PhD PhD in Nuclear
Engineering (Universite' Libre de Bruxelles, 2005). 
   His group is actively pursuing research in
the propagation of
uncertainty in data by means of the Total Monte-Carlo method and the
development of Reduced Order Models to speed-up the application of the former.
  This team will contribute to the proposal by providing relevant information
on the MYRRHA safety analysis, more specifically on those transients where
decay heat removal and the corresponding systems play a crucial role. Also a
close collaboration in the frame of the current PhD research on ROM development
for MYRRHA will lead to mutual benefit.  
