
\vspace*{.1in} \noindent \textbf{Detailed Work Plan.}
We turn now to the principal ingredients from \cite{fick18} that need
to be developed to have a systematic analysis tool.  These components
will be required for each of the Tasks listed above.

The equations of interest are the incompressible Navier-Stokes equations
(NSE) with energy transport
\begin{eqnarray} \label{eq:ns}
&&
\partial_t \bu \,+\, \bu \cdot \nabla \bu \,-\, \frac{1}{Re} \nabla^2 \bu
\,+\, \nabla p \;=\; \bff, \hspace*{.4in} \nabla \cdot \bu \;=\; 0, 
\\ \label{eq:therm} 
&&
\partial_t T   \,+\, \bu \cdot \nabla T   \,-\, \frac{1}{Pr\,Re} \nabla^2 T
\;=\; q, 
\end{eqnarray}
subject to appropriate boundary and initial conditions. Here, $Re$ and $Pr$ are
respective Reynolds and Prandtl numbers, $q$ is the volumetric heating, and
$\bff$ is a body force that may be temperature dependent.
   To simplify the exposition, we here consider only the NSE (\ref{eq:ns}).

One of the distinguishing features of turbulent flows is that, because of
sensitivity to initial conditions, they are repeatable only in a statistical
sense.  There is no hope of repeating the exact trajectory of the solution in a
turbulent flow between two experiments or two simulation algorithms.  ROMs,
therefore, must target {\em space-} and/or {\em time-averaged} QOIs such as
velocity profiles,  turbulent-kinetic energy distributions, and Nusselt
numbers.  For the same reason, such quantities are also the currency of
engineering design and thus reasonable measures of the success of a ROM as a
low-cost analysis tool.  Thus, the {\em objective} of this project is
reproduction of engineering QOIs, rather than precise reproduction of
trajectories.  This simplified objective is one of the key features that makes
the problem tractable.

    While our focus will be on mean engineering quantities, we note that the
time-evolving ROMs {\em also} provide reasonable surrogates for unsteady
behavior (or they would not be good predictors of the mean) and therefore can
be used to visualize flows, which is often useful for engineering insight
(e.g., identifying stagnant and ejected vortices).  We also propose to augment
qualitative visual analysis of the unsteady behavior with quantitative
monitoring of the amplitudes, frequencies, and gradients of QOIs at specified
locations in subsets of the domain, which might help to evaluate flow induced
vibration, thermal striping, thermal fatigue, and thermal stratification
effects.

For each Task, we break the ROM development into a {\em reproduction problem}
and a {\em parametric problem}.
  The reproduction problem amounts to building a ROM capable of reproducing
the QOIs to a desired accuracy at a given point in $\cP$.
  Solution of the parametric problem entails generation of ROMs at several
optimal points in a finite-dimensional training set, $\cP_{\mbox{train}}
\subset \cP$, chosen to minimize the number of offline simulations for a given
error tolerance.
  The reproduction problem effectively reduces the spatial dimension (through
the POD) and time dimension (through the time-averaged QOIs), while the
parametric problem addresses reduction of the parameter space.
The following sections give a brief overview of these two key components.


