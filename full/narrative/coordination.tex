
%% {\tt
%%  • The roles and responsibilities of each partnering organization
%%  in  the execution of the workscope. Describe the role and work to
%%  be  performed by each participant/investigator, the business
%%  arrangements  between the applicant and participants, and how the
%%  various efforts  will be integrated and managed.
%% }

\vspace{.05in}

%%  \noindent
%%  The project team comprises university partners Fischer (UIUC) and Merzari (PSU)
%%  and laboratory partner Shaver (ANL), all working to develop ROM analysis tools
%%  that are readily usable by researchers from industry, academia, and national
%%  laboratories.  
%%  
%%  Fischer will lead the investigation of advanced ROMs for turbulent flow
%%  simulation along with error estimator developments.  His team will focus on
%%  cost reduction through optimized algorithms and parallel implementations for
%%  both the offline and online analysis.  The team will interact with the other
%%  members in identifying anomalous behavior, analyzing the underlying
%%  implementations, and verifying the physical relevance of the ultimate outcomes.
%%  
%%  Merzari will lead validation/verification efforts in identifying and performing
%%  full-order/reduced-order simulations appropriate for the ultimate target
%%  application of SFR LOF and transient analyses.  This effort will include
%%  careful setup of the FOMs for the validation problems and identification of
%%  relevant QOI's.   Merzari and Shaver will work on integration of NekROM with
%%  Cardinal for system analysis modules (SAMs).   Fischer, Merzari, and Shaver
%%  will work together on the coupled LES/ROM development.
%%  
%%  All team members will have access to project software updates,  maintained
%%  through a \texttt{git} repository, further coordinated with \texttt{github}.  The
%%  team will have biweekly video conferences and semi-annual face-to-face meetings
%%  (coordinated with other meetings that are already scheduled), to facilitate
%%  review and planning for the next stages of the project.
%%    Publications in peer-reviewed journals will consolidate the principal
%%  results.  Validated pMOR examples and software updates will be made available
%%  to the reactor design community through the open-source NekROM repository
%%  (which will soon be part of the Nek5000/RS git repository).


\noindent
The project team comprises university partners Fischer (UIUC) and Merzari (PSU)
and laboratory partner Shaver (ANL), all working to develop ROM analysis tools
that are readily usable by researchers from industry, academia, and national
laboratories.  PSU and ANL are subcontractors on the UIUC award.

The team will meet every two-weeks to coordinate on different development
aspects of the project. Detailed work on specific models will be developed by
each co-PI together with the graduate students who will be supported by this
project. Each collaborating person/institution contributes a complementary set
of expertise, and their responsibilities will be as follows: 
\begin{itemize}
    \item \textit{Paul  Fischer} will  lead the overall project and ensure a
smooth and productive interaction among all areas. He will also lead overall
algorithmic development with a focus on order reduction.  

\item \textit{Elia Merzari} will lead the high fidelity simulation thrust
for Large Eddy SImulation (LES) and Direct Numerical Simulation (DNS) datasets
in rod bundles as well as the validation, testing and demonstration thrust. He
will also support algorithmic development by focusing on the interaction
between high fidelity and the reduced order models.

\item \textit{Dillon
Shaver} will support ongoing efforts by consulting on the relevance to the
NEAMS program and the advanced reactor industry.  He will also work with Merzari
on directing the integration of NekROM with Cardinal/SAM for target cases.
\end{itemize}

