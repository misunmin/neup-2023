
% This section should be formatted to address each of the merit 
% review criterion and sub-criterion listed in Part V, Section A. 

% > Provide sufficient information so that reviewers will be able 
%  to evaluate the application in accordance with these merit review 
%  criteria. 

% > DOE has the right to evaluate and consider only those applications
%   that separately address each of the merit review criteria.

%O Criterion 1: Advances the State of Scientific Knowledge and Understanding and
%O              Addresses Gaps in Nuclear Science and Engineering Research 35% 
%O 
%O Criterion 2: Technical Quality of the Proposed R&D Project 35%
%O 
%O Criterion 3: Applicant Team Capabilities, Risks, and Experience 30%

\subsubsection*{Response to Merit Review Criteria}
The proposed project effectively addresses the following merit review
criteria for this program.

\noindent \textbf{1.}
  The project will advance the state of knowledge in parametric model order
reduction (pMOR) for turbulent flows in general and for SFR applications in
particular.  Practical deployment of pMOR requires rigorous analysis to develop
effective error indicators; a deep understanding of the underlying physics; and
careful algorithm development to ensure speed and correctness of the FOM
and ROM.  We are particularly focused on ROMs for challenging large-scale
turbulent flows, which remains an open research question, but one that should
be tractable given current-day computing resources and novel developments in
reduced-order modeling.  The work will equip researchers in reactor design and
other thermal analysis fields with advanced modeling capabilities.

\noindent \textbf{2.}
The technical quality of the project is high.  It combines rigorous
mathematics, practical numerical analysis, and state-of-the-art algorithms to
significantly extend the capabilities of the scalable high-order code
Nek5000/RS on the Nation's leading-edge computing facilities, including DOE's
exascale platforms. (NekRS has recently met a critical Figure of Merit
milestone as part of the DOE's ExaSMR project, which is one of only three
applications to meet the FOM to date.)  pMOR/ROM provides a mechanism to
leverage these and future computers for effective engineering analysis of
challenging problems in reactor thermal hydraulics, with a particular 
emphasis on mixed convection and long-time transients in SFRs.

\noindent \textbf{3.}
The applicant team is highly experienced and capable of successfully completing
this project.  The team comprises experts in reactor thermal-hydraulics, CFD,
applied mathematics, numerical algorithms, turbulence, and heat transfer.  The
PI is the lead developer of Nek5000 and a co-developer of NekRS.  Merzari and
Shaver lead the integration of Nek5000/RS into the NEAMS tool suite for reactor
analysis.  As a team, the PIs have co-authored more than 30 articles in reactor
thermal-hydraulics, with many of these focused on ROMs.  The PI and his
students at UIUC continue to push ROM development for turbulent flows, with
significant advances realized in augmented basis methods (Fig. \ref{fig:abm}
$\,$ \cite{kaneko22a}) and in feasibility of ROMs for buoyancy-driven flows
\cite{kaneko20a,tsai22a}.  Merzari has been a pioneer in exploring POD-based
ROMs for TH applications.  
%% The team is also collaborating with the Virginia Tech group of Traian Iliescu
%% to push advanced developments in ROMs for turbulence \cite{mou21}.

Potential risks and mitigation strategies for the project are identified in the
Unique Challenges section.


