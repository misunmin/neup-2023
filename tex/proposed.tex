\noindent
{\bf Proposed Tasks.}
We propose to extend Nek5000/RS and NekROM to support a multiscale simulation
environment.   The first set of tasks will be to provide more seemless integration
between Nek5000/RS and NekROM.  Ideally, one should be able to generate a ROM
on the fly.



We propose the following tasks to realize a multiscale simulation
capability for reactor analysis:
\begin{description}
%
\item{\em i.}
Continued development of ROM closure models for increased robustness when
simulating high Rayleigh/Reynolds flows.  This effort will build on the work of
\cite{kaneko22a,kaneko22,tsai22a} to extract more accuracy and detail out of
stabilized ROMs for turbulent flows.
%
\item{\em ii.}
Identify strategies to equip the ROM for long-time transients.  While progress
was demonstrated in \cite{kaneko20a}, more needs to be done to make the
approach robust, particularly using alternative basis functions and error
indicators.
%
\item{\em iii.}
Develop multi-scale methods that couple ROMs with LES on a subset of
the domain.  Principal challenges are to support general or
parameterized inflow conditions for the ROM and to quantify relaxation
distances over which small scale structures will evolve in the LES region
near the ROM-LES interface.
(These distances are often short in complex domains.)
The goal will be to combine reactor-scale ROM simulations with localized
detailed simulations in regions of interest.
%
\item{\em iv.}
Validate the accelerated transient calculations with high-fidelity counterparts
in an SFR rod-bundle at low-flow conditions.
%
\item{\em v.}
Perform ROM-based parameter sweeps and compare these to corresponding
FOM results for validation.
%
\end{description}


%-----------------------------------------------------------------------------


















