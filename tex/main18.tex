

%\section{Project Narrative}
%% Applicant shall provide a written narrative addressing the strategy
%% to execute R&D that supports the specified Technical Workscope. The
%% documentation provided shall include the items specified below:

\vspace*{.0in} \noindent 
\underline{\textbf{Title of Project}} 
\\[-2ex]

\noindent
\textbf{Multiscale Modeling for Reactor Thermal Hydraulics


\vspace*{.15in} \noindent 
\underline{\textbf{Technical Workscope Identification:  NEAMS-1.4}}
\\[-2ex]


\vspace*{.05in} \noindent 
\underline{\textbf{Project Objectives.}} 
\\[-2ex]

  The objective of this project is to develop a reduced-order model for the
simulation of stratified flows characteristic of the upper plenum in liquid
metal reactors.  The code suite will translate high-fidelity large-eddy
simulations of turbulent stratified flow into a set of coefficients that can
drive a low-dimensional dynamical system relating output to input quantities
(e.g., temperatures and flow rates in a multiport plenum) suitable for 
systems analysis codes.  
\\[0ex]

\vspace*{.0in}\noindent \underline{\textbf{Proposed Scope Description}}%  (< 1 page )
\\[-2ex]


The scope of the proposed work is to develop an effective {\em analysis chain}
that will allow designers to accurately predict thermal-hydraulics behavior
over the broad range of temporal and spatial scales that are relevant to overall
reactor behavior.
    The tools to tackle this multiscale phenomena include NekRS, which is 
a GPU-variant of Nek5000 that is designed to fully leverage the performance of
DOE's leadership computers, including the exascale platforms Frontier and
Aurora; and NekROM, which is a new reduced-order modeling module within
Nek5000/RS that was developed under NEUP Award 18-15520.






{\em reduced-order
models} (ROMs) for reactor thermal design and analysis with the specific target
problem of the upper plenum in a liquid metal reactor (LMR).  Characteristics
of flow in the upper plenum include multiple inlets at various temperatures, a
stratified background, natural circulation, and multiple outlets.  
   ROMs typically comprise a small system of ordinary differential equations
(ODEs) that represent the evoluton of $N \approx 100$ basis functions designed
to capture the dominant dynamics of the flow field.
   {\em Parametric model order reduction} (pMOR) is a process through which
a ROM or suite of ROMs is constructed to estimate the system behavior 
over a range of parametric inputs (e.g., inlet flow rates, Reynolds number)
without the need for high-fidelity simulations at every point in the target
parameter space.  The output of the pMOR will be a small set 
of coefficients to drive low-rank systems of ODEs that can be directly coupled
with into the Aystems Analysis Module (SAM) to provide a more accurate response
than currently-used zero-dimensional models for the upper plenum.

The ROMs will be derived from high-fidelity direct numerical
(DNS), large-eddy (LES), or unsteady Reynolds-averaged Navier-Stokes (uRANS)
simulations of turbulent thermal transport, which will performed on DOE's
leadership computing facilities in an {\em offline} mode using the
NEAMS-supported Nek5000 code.  The ROMs will be used in a real-time {\em
online} mode (e.g., in SAM), simulating the governing Navier-Stokes and energy
equations as a projection of a small set of basis functions derived from a
proper-orthogonal-decomposition (POD) of snapshots generated by the
high-fidelity simulations.  Innovative features of the proposed work include a
stabilized POD-Galerkin formulation based on constrained optimization,
efficient algorithms for basis selection to minimize the size, $N$, of the
reduced basis set;  and a rapidly convergent greedy training scheme using
dual-norm based error indicators to provide accurate analyses at low cost.  
Verification and validation (V\&V) will be performed by comparing the ROM
against DNS/LES-based full-order models (FOM) {\em and} against available
experimental data relevant to the upper plenum (e.g., \cite{lomperski17}).


The proposed analysis tool requires 
 \textbf{i.} detailed full-order simulations of size (number of degrees-of-freedom)
$\cN=10^7$--$10^{11}$;
 \textbf{ii.} a software module within Nek5000 to distill each full-order
simulation into ROMs of size $N \approx 100$;
 \textbf{iii.} a software driver to iterate between the ROM and Nek5000 to 
pick optimal anchor (trial) points in the target parameter 
(e.g., Reynolds or Prandtl number) space, $\cP$;
and
 \textbf{iv.} a simulation code to evolve the ROM in the online mode. 
The principal focus will be on developing ( \textbf{ii.}-- \textbf{iv.}), supported by
simulations with Nek5000 (\textbf{i.}), which is well-established as a scalable
code for solving the full-order models (e.g., \cite{merzari11b,merzari15a}). 

Realization of a viable analysis tool requires economization of costs at all
stages of the process, including a reduction in the number of expensive
offline simulations and a reduction in the size of the
reduced-basis sets $N$, which incur costs scaling as $O(N^3)$ because 
of the nonlinear interactions in the Navier-Stokes equations.
In addition to costs, stability and accuracy of the ROM are also critical.

  To address these concerns, the proposed methodology will expand upon recent
developments in ROMs for long-time simulations of turbulent flows outlined in
\cite{fick18}.  Key ingredients of this approach are:
 \textbf{i.} $H^1_0$-POD-based basis sets for unsteady turbulent flows
         at fixed anchor points $\cP_{\mbox{\tiny anch}}$ in the target
         parameter space, $\cP$;
 \textbf{ii.} a novel constraint-based stabilization of the ROM derived
          from the full-order models;
 \textbf{iii.} effective dual-norm error estimates that {\em inexpensively}
           indicate the model error at any point in $\cP$; and
 \textbf{iv.} an efficient greedy algorithm for optimizing the choice of
          anchor points to meet the desired model error tolerance 
          for any point $\up^* \in \cP$.
We remark that the error estimators are essential to an efficient POD
formulation of this problem and are a {\em unique} contribution of the proposed
approach.  The constraint-based stabilization of the ROM is also {\em unique}
to the proposed effort.  This project will provide the first tests of this
approach in industrially-relevant settings.   

\vspace*{.08in}\noindent \underline{\textbf{ Logical Path to Accomplishing Scope.}} \\[-2ex]

As part of the development effort, the project will address several NE-relevant
test cases for validation of the overall modeling process.  We will start by
extending the 2D unsteady lid-driven cavity problem of \cite{fick18} to support
the energy equations.  We will next consider forced turbulent convection in a
T-junction, for which significant Nek5000 simulation and POD models have
already been validated against experimental data \cite{merzari11b}.
The final extension to address buoyancy-driven flows, such as arise in
the upper plenum
of Liquid Metal Reactors (LMRs) \cite{mbd1,mbd2,mbd3} based on available test data
as in \cite{mbd4} and prepare an input for system analysis modules (SAMs).  This
effort will include careful setup and validation of the full-order Nek5000 models 
for the target upper plenum simulations and identification of relevant 
upper plenum QOI's needed for the Systems Analysis Module.
\\[-1ex]



The proposed work has been broken into four tasks to be carried
out over the three-year project period: \\[-4.0ex]
\begin{itemize}
\item Task 1:  Extension of \cite{fick18} to 3D turbulent flows.
\\[-4.3ex]
\item Task 2:  Extension to the energy equation and thermally-driven QOIs.
\\[-4.3ex]
\item Task 3:  3D turbulent flows with thermal transport.
\\[-4.3ex]
\item Task 4:  Extension to buoyant 3D turbulent flows.
\end{itemize}
Task 1 is a logical first extension to the methodology described in \cite{fick18},
which used the unsteady 2D lid-driven cavity as a test problem.  
Task 2 incorporates an additional equation and thus requires additional
theory to establish appropriate error indicators for QOIs relevant to thermal
transport.
Task 3 combines the results from Tasks 1 and 2 for a first trial of
the method for turbulent thermal transport.
Task 4 incorporates additional (linear) coupling in the ROM methodology.

Each Task $\tau$ has the following subtasks, \\[-4ex]
\begin{itemize}
\item %Task $\tau$:
%\\[-3.6ex]
   \begin{itemize}
     \item Subtask $\tau$.1: Set up/execute DNS/LES full-order model in Nek5000
\\[-3.6ex]
     \item Subtask $\tau$.2: Formulation of error indicators for the problem.
\\[-3.6ex]
     \item Subtask $\tau$.3: Identify POD requirements (e.g., number of input
                       snapshots and size of the POD basis) for the
                       {\em solution reproduction problem}.
\\[-3.6ex]
     \item Subtask $\tau$.4: Build/extend the offline/online ROM 
                       modules to accommodate the new physics.
\\[-3.6ex]
     \item Subtask $\tau$.5: Verification/validation tests for the {\em parametric problem}. \\[-4ex]
   \end{itemize}
\end{itemize}
We note that verification/validation ($\tau$.5) is relatively straightforward in
the ROM setting given that one can run a full-order model at any test point in
a valid parameter range $\cP$.  Moreover, the sequence of Tasks,
$\tau$=$1,\dots,4$ is part of the validation process.  Each Task is an incremental
step between the current state of the art of 2D unsteady ``turbulence'' to the
target of 3D buoyancy-driven turbulence.  This systematic approach allows 
identification of potential pitfalls along the development path.

The effort requires correct problem specifications (parameters, boundary
conditions, mesh resolution, etc.) and efficient set-up of the large-scale
full-order models (FOMs) on DOE's leadership computing facilities.  We
anticipate acquiring computer time through DOE's standard award mechanisms
(director's discretionary, ALCC, or INCITE) and leveraging other resources
already available to the team members.  The ROM-modeling will involve extension
of the 2D unsteady Navier-Stokes setting in \cite{fick18} to 3D turbulent flows
with energy transport.  The error estimators will be modified to account for
quantities of interest (QOIs) relevant to the upper plenum flow, such as mean
and RMS temperature distributions, mean outflow distributions, etc.
Insight to temporal behavior will also be accessible from visualizations
of the ROM reconstructions.

\input narrative/plan

\input narrative/rom

\input narrative/merit

\vspace*{.05in}\noindent \underline{\textbf{Relevance and Outcomes/Impacts}} \\[-2ex]
\input narrative/relevance

%% • Relevance and Outcomes/Impacts: This section will explain the 
%%   program relevance/priority of the effort to the objectives in the 
%%   program announcement and the expected outcomes and/or impacts.


\vspace*{.05in}\noindent \underline{\textbf{Schedule}} \\[-2ex]
\input narrative/schedule 

\vspace*{.05in}\noindent \underline{\textbf{Milestones and Deliverables}} \\[-2ex]
\input narrative/milestones 

\vspace*{.05in}\noindent \underline{\textbf{Facilities}}
% • Type/Description of facilities that will be used to execute the
%   scope (if applicable).

\input narrative/facilities

\vspace*{.05in}\noindent \underline{\textbf{Roles/Responsibilities of 
Partnering Organizations}}

\input narrative/coordination


\vspace*{.05in}\noindent \underline{\textbf{Unique Challenges}}
% Unique challenges to accomplishing the work and planned mitigations.

\input narrative/challenge

