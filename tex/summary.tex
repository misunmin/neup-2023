

\subsection{High-Fidelity Simulations}

Nek5000/RS is a high-scalable open-source code designed for the simulation
of turbulence in thermal-fluids problems.   The 

USUAL Nek5000/SEM stuff here.
Mention the term HPC


Most recently, NekRS, a GPU variant of Nek5000, has been developed to
exploit the DOE's forthcoming exascale platforms, all of which use
accelerator-based nodes where the majority of the compute power derives
from GPUs or similar fine-grained SIMD parallel architectures.
  What NekRS does w.r.t. performance
Because it uses the same algorithms as Nek5000, NekRS inherits
all of the advantages of the SEM formulation, the advanced 
time-stepping strategies, and the scalable linear solvers.
In fact, many of the solvers in NekRS are currently more advanced
than those in Nek5000.   Among these are recently-developed variants
of $p$-multigrid with Chebyshev-accelerated Schwarz smoothers \cite{siam22a}.

Show some scaling plots (Crusher, etc.)

\subsection{Data-Driven Reduced-Order Models}

(WHY ROM)
While it is clear that NekRS will be capable of delivering fast turn-around
for reactor-scale simulations on DOE's exascale platforms, the use of 
such simulations for one-off determination of system behavior at a single
parameter point does not constitute an efficient utilization of resources. 
It far more effective from a design perspective if one can reliably explore
parametric input/output relationships (e.g., Nusselt/Rayleigh-number in
low-flow conditions) in the neighborhood of the parameter space for which
the high-fidelity DNS or LES is performed.   Such a capability is precisely
the goal of parametric model-order reduction (pMOR) which is typically based
on reduced-order models (ROMs) of the high-fidelity DNS/LES (often referred
to as full-order models, or FOMs).   

ROMs are typically built by collecting snapshot fields of the FOM solution


For unsteady turbulent flows, ROMs need to address two problems: 
{\em (i), the reproduction problem} in which the 




(HOW ROM)











