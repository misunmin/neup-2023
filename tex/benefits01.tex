
\parindent 0in
\parskip 0.1in

\begin{center}
{\bf Benefits of Collaboration } \\
{\it  Reduced Order Modeling of heat and fluid flow: Multi-scale modeling of advanced reactors to enable faster deployment }
\end{center}

The collaborative team assembled for the proposed work consists of nuclear
engineers, mechanical engineers and computer scientists.  The collaborators
cover the entire spectrum of knowledge necessary to pursue the proposed work.

To the PI's knowledge,
this proposal has no benefit or influence on other proposed NE projects.

\section{Principal Investigators: Short Biographies}
\begin{itemize}
\item \textit{Paul Fischer} (UIUC) is a Blue Waters professor in the
departments of Computer Science and Mechanical Science \& Engineering
at the University of Illinois, Urbana-Champaign.  He is the lead developer
of the open source thermal-fluids code, Nek5000, which is a Gordon Bell
and R\&D 100 Prize winning code.  Fischer is a pioneer in high-order
numerical methods and scalable parallel algorithms for scientific computing
and has designed (and tested) Nek5000 to run on over a million processes.
He works closely with scientists to address challenging problems over a broad
spectrum of applications ranging from biofluids to astrophysics.
Under prior NEUP funding, he and his students have made significant advances
in the development of reduced-order models for turbulent flow simulations
relevant to reactor analysis.



\item \textit{Elia Merzari} (PSU), is an Associate professor in nuclear
engineering and associate editor of Nuclear Engineering and Design, and
previously held various leadership roles as a principal nuclear engineer at
Argonne National Laboratory (ANL). His expertise spans nuclear engineering
multi-physics applications , high-performance computing, CFD simulations, and
reduced order models as well as NEUP-funded nuclear engineering work.  He has
worked for 15 years in the area of Proper Orthogonal Decomposition, a key
method for this proposal.

\item \textit{Dillon Shaver} (ANL) is a Principal Nuclear Engineer in the
Nuclear Science and Engineering Division at Argonne National Laboratory, and
responsible for the high fidelity thermal-hydraulic simulation in the Nuclear
Energy Advanced Modeling and Simulation (NEAMS) program. He has worked
extensively in the area of Computational Fluid Dynamics and he has been
instrumental in developing multiphase and Reynolds Averaged Navier Stokes
models in Nek5000. He has broad experience in the use of high fidelity models
in the advanced reactor industry.

\end{itemize}
\section{Principal Investigators: Interaction and Responsibilities}

The team will meet every two-weeks to coordinate on different development
aspects of the project. Detailed work on specific models will be developed by
each co-PI together with the graduate students who will be supported by this
project. Each collaborating person/institution contributes a complementary set
of expertise, and their responsibilities will be as follows: 
\begin{itemize}
    \item \textit{Paul  Fischer} will  lead the overall project and ensure a
smooth and productive interaction among all areas. He will also lead overall
algorithmic development with a focus on order reduction.  

\item \textit{Elia Merzari} will lead the high fidelity simulation thrust
for Large Eddy SImulation (LES) and Direct Numerical Simulation (DNS) datasets
in rod bundles as well as the validation, testing and demonstration thrust. He
will also support algorithmic development by focusing on the interaction
between high fidelity and the reduced order models.  

\item \textit{Dillon
Shaver} will support ongoing efforts by consulting on
the relevance to the NEAMS program and the advanced reactor industry.
\end{itemize}
\section{Synergistic Acitvities}
The work will also have a direct impact on NEAMS, with applications of direct
relevance to the nuclear industry, and indirectly to other DOE applications
that rely on fluid flow and heat transfer.The project has the potential to
enable a transformative change in the way thermal-hydraulic simulations are
performed.

Paul Fischer is the deputy director of the DOE Center for Efficient Exascale
Discretizations (CEED), which is supporting the development of NekRS,
the exascale-ready version of Nek5000.  As lead developer for Nek5000, he
co-directs the NekRS development to ensure that it meets expected performance
goals and the needs of the Nek5000 users community.  The proposed work will
continue to drive NekRS development and foster expansion of the user base
and capabilities of the reduced-order modeling toolkit, NekROM.

Elia Merzari is currently leading a NEAMS IRP developing a multiscale framework
capable of using high-resolution simulation data to improve low-order models
through better closures. This effort ties into the proposed work, and lessons
learned from that effort will be leveraged as it is a natural evolution of that
work.

The PIs are active in the nuclear thermal-hydraulics community and the American
Nuclear Society in particular (e.g., Elia Merzari is the immediate past chair
of the thermal-hydraulic division and Dillon Shaver is the current Program
chair). Results from the project will be disseminated in conferences, special
sessions and journal articles.


\section{Facilities}
The proposed computational work will be performed at various high performance
computing facilities, where the PIs already have been awarded allocations.

The PI has access to {\em Delta} at UIUC, which has 440 NVIDIA A100 GPUs.  His
lab also has several multi-CPU workstations, including one with two NVIDIA
Titan V GPUs for local development work.  Through the DOE ECP CEED project, he
and his students have access to DOE's pre-exascale and exascale platforms for
testing at scale.  He also frequently has access to ALCF and OLCF platforms
through DOE's standard allocation processes.


The co-PI at Penn State has an allocation at the local ROAR supercomputer that
will be leveraged for both testing and optimization simulations. ROAR is Penn
State’s flagship research cluster maintained by ICDS. Roar includes basic,
standard, and high memory compute as well as GPU processors. ROAR operates more
than 30,000 Basic, Standard and High Memory cores to support Penn State
research. The system provides dual 10- or 12-core Xeon E5-2680 processors for
Basic and Standard memory configurations and quad 10-core Xeon E7-4830
processors for High Memory configurations. In addition to access to ROAR,
request for allocations with DOE’s ALCC and INCITE programs will be submitted
for access to DOE’s supercomputing facilities if needed. Access to INL Sawtooth
will also be pursued.
%
% {\bf D.2 Benefit of Collaboration} (4 pages)
%
% Applicant shall provide a narrative that includes an explanation of the
% contribution that will be made by the collaborating organizations and/or
% facilities to be utilized. It may contain brief biographies of staff and
% descriptions of the facilities wherein the research will be conducted. Please
% indicate within this section whether the application has benefit or influence
% on other ongoing or proposed NE R\&D projects (e.g., modeling and simulation in
% one application and effect validation in a separate application).
%
% This document is required unless the application only has a single principal
% investigator.
%
% Pages outside the specified page limits and font size, including references,
% will be redacted and unavailable for evaluators to review.
%
% o 4-page limit, 11-point font.
%
% Name File: 2023 RPA Benefit of Collaboration “Insert ID \#”
